\documentclass[12pt]{article}

% --- Packages ---
\usepackage{amsmath, amsthm, amssymb}
\usepackage{graphicx}
\usepackage{geometry}
\usepackage{fancyhdr}
\usepackage{enumerate}
\usepackage{listings}
\usepackage{xcolor}
\usepackage{setspace}
\usepackage[utf8]{inputenc}
\usepackage[english]{babel}
\usepackage{framed}
\usepackage{caption}
\usepackage{booktabs}     % for \toprule etc.
\usepackage{siunitx}      % for \SI, units
\usepackage{array}
\usepackage{makecell}     % for multi-line table cells
\usepackage{tabularx}     % auto-wrapping column type X
\sisetup{detect-all, per-mode=symbol} % nicer units
\usepackage{enumitem}

% Keep hyperref LAST
\usepackage{hyperref}

% ragged-right X column that still allows \\ at row ends
\newcolumntype{Y}{>{\raggedright\arraybackslash}X}


% Page Layout
\geometry{
    textheight=9in,
    textwidth=6.5in,
    top=1in,
    headheight=12pt,
    headsep=25pt,
    footskip=30pt
}

\begin{document}
\begin{center}
    {\LARGE \textbf{Ultrasonic Sensor Timing Sequence}}\\[0.5em]
\end{center}
\begin{figure}[h!]
    \centering
    \includegraphics[width=0.95\linewidth]{ping_timing_diagram.png}
    \caption{\small PING))) Ultrasonic Sensor Timing Sequence (Single-Sensor Cycle).}
    \label{fig:ping_timing}
\end{figure}

\small

\noindent One full \SI{20}{ms} measurement cycle for a single \textbf{Parallax PING))) Ultrasonic Distance Sensor} via an \textbf{SN74LVC245A} bidirectional level shifter. The MCU issues a \SI{5}{\micro\second} trigger, then reconfigures the shared I/O to input to measure the echo pulse width (time-of-flight).

\begingroup\scriptsize
\vspace{4pt}
\begin{center}
\renewcommand{\arraystretch}{1.2}
\begin{tabularx}{\linewidth}{@{} l l Y @{}}
\toprule
\textbf{Signal} & \textbf{Name / Direction} & \textbf{Description} \\
\midrule
\textbf{DIR} & Buffer Direction &
\textit{SN74LVC245A} direction control.\newline
1 = MCU $\rightarrow$ Sensor (trigger phase).\newline
0 = Sensor $\rightarrow$ MCU (echo phase).\\[3pt]
\hline
\textbf{PING[$i$]} & Sensor GPIO (Group) &
Bidirectional MCU GPIO shared with ultrasonic sensor $i$.\newline
Logical group encompassing IMASK, DOE, TRIG, and ECHO signals.\newline
Manages the trigger, echo, and interrupt states for each sensor in the round-robin sequence.\\[3pt]
\hline
\textbf{IMASK} & Interrupt Mask &
Per-pin interrupt enable for the PING line.\newline
1 = Interrupt enabled (ECHO edge capture active).\newline
0 = Interrupt disabled (no capture events forwarded to NVIC).\\[3pt]
\hline
\textbf{DOE} & Data Output Enable &
MCU GPIO output-enable control for the PING line.\newline
Input: idle and echo phases (interrupts disabled during idle).\newline
Output: trigger phase.\\[3pt]
\hline
\textbf{TRIG} & Trigger Pulse &
\SI{5}{\micro\second} rising-edge pulse driven by MCU to initiate ultrasonic burst from the sensor.\\[3pt]
\hline
\textbf{ECHO} & Echo Pulse &
Return pulse width proportional to target distance.\newline
Minimum: \SI{115}{\micro\second}\newline
Maximum: \SI{18.5}{\milli\second}.\\[3pt]
\hline
\textbf{TIMER} & Capture / Compare &
Timer capture/compare reference points (ZERO, CC0, CC1, LOAD) aligned with echo pulse duration measurement.\\
\bottomrule
\end{tabularx}
\end{center}
\endgroup

\noindent\textbf{Notes.}
\begin{enumerate}
  \item Timer in edge-aligned up-count mode (LOAD = period limit).
  \item Preload GPIO output before asserting DOE to avoid transients.
  \item Round-robin across sensors with \SI{20}{ms} spacing to avoid ultrasonic crosstalk.
  \item SN74LVC245A provides \SI{5}{V} $\leftrightarrow$ \SI{3.3}{V} logic compatibility.
\end{enumerate}

\newpage
\begin{center}
    {\LARGE \textbf{Multi-Sensor Schedule Cycle}}\\[0.5em]
\end{center}

\begin{figure}[h!]
    \centering
    \includegraphics[width=0.95\linewidth]{ping_round_robin_diagram.png}
    \caption{\small Round-robin schedule for three ultrasonic sensors.}
    \label{fig:ping_round_robin}
\end{figure}

\small
\noindent The system operates three ultrasonic sensors in a round-robin sequence to prevent acoustic interference between transducers. Each sensor is allocated a dedicated \SI{20}{ms} time slot encompassing its trigger, echo measurement, and a brief guard interval before the next sensor’s cycle begins. This ensures that no two sensors emit ultrasonic bursts simultaneously.

\vspace{6pt}
\begin{center}
\renewcommand{\arraystretch}{1.15}
\begin{tabular}{ccc}
\toprule
\textbf{Slot} & \textbf{Time Window} & \textbf{Active Sensor} \\
\midrule
0 & $[\,\,\,\,\SI{0}{ms},\,\SI{20}{ms})$   & PING[0] \\
1 & $[\,\SI{20}{ms},\,\SI{40}{ms})$  & PING[1] \\
2 & $[\,\SI{40}{ms},\,\SI{60}{ms})$  & PING[2] \\
\bottomrule
\end{tabular}

\vspace{3pt}
\footnotesize
Cycle repeats every \SI{60}{ms}, yielding an update rate of \SI{16.7}{Hz} per sensor.
\end{center}
\vspace{6pt}
\noindent\textbf{Timing considerations.}
As shown in the single-sensor timing diagram, each measurement cycle consists of a short trigger, a fixed holdoff interval, and an echo pulse lasting up to approximately \SI{18.5}{ms}. Including these intervals yields a total cycle time of about \SI{19.2}{ms}, so a \SI{20}{ms} slot provides a small but sufficient margin to accommodate overhead and ensure all echoes have dissipated before the next sensor begins.\\
\\
\noindent\textbf{Notes.}
\begin{enumerate}
  \item Only one sensor is active per slot; the remaining sensors stay configured as inputs with their internal pull-downs enabled to prevent bus contention and maintain a defined low idle state. 
  \item The MCU cycles through sensors sequentially every \SI{20}{ms}.
  \item This schedule avoids cross-interference while maintaining a \SI{50}{Hz} global timing base.
  \item Additional sensors can be supported by proportionally extending the total cycle period.
\end{enumerate}

\newpage
\begin{center}
    {\LARGE \textbf{Differential-Drive Odometry}}\\[0.5em]
\end{center}
\noindent
\textbf{Odometry} refers to estimating a robot’s position $(x, y)$ and orientation $\theta$ over time by integrating wheel encoder measurements. 
In a differential-drive system, changes in left and right wheel displacements determine the robot’s forward motion and rotation.

\begin{table}[h]
\centering
\scriptsize
\renewcommand{\arraystretch}{1.15}
\begin{tabular}{@{}llp{8.5cm}@{}}
\toprule
\textbf{Symbol} & \textbf{Name / Role} & \textbf{Definition / Notes} \\
\midrule
$N_L,\;N_R$ & Encoder counts (cum.) & Cumulative quadrature counts (ticks) for left/right wheels. \\
$\Delta N_L,\;\Delta N_R$ & Encoder count increments & Counts accumulated since the last update step (ticks). \\
$T$ & Ticks per revolution & Encoder ticks per wheel revolution (include quadrature mode and gearbox). \\
$R$ & Wheel radius & In meters. Used to convert ticks $\to$ distance. \\
$b$ & Wheel track & Lateral distance between wheel contact patches (m). \\
$(x,\;y,\;\theta)$ & Robot pose & World-frame position and heading; $\theta$ measured CCW from $+x$ (radians). \\
\midrule
$\Delta s_L,\;\Delta s_R$ & Wheel travel (step) & Left/right wheel arc lengths over the last step:
$\displaystyle \Delta s_L=\frac{2\pi R}{T}\,\Delta N_L,\quad
\Delta s_R=\frac{2\pi R}{T}\,\Delta N_R.$ \\
$\Delta s$ & Chassis forward travel (step) & Body-frame arc length over the last step:
$\displaystyle \Delta s=\tfrac{1}{2}(\Delta s_R+\Delta s_L).$ \\
$\Delta \theta$ & Heading change (step) & Turn over the last step:
$\displaystyle \Delta \theta=\frac{\Delta s_R-\Delta s_L}{b}.$ \\
$s$ & Path length (cumulative) & Total forward distance traveled along the chassis arc. Update each step by
$\displaystyle s \leftarrow s+\Delta s.$ (Some notes use $s$ for a single step; here $s$ is cumulative and $\Delta s$ is per step.) \\
\bottomrule
\end{tabular}
\end{table}

% Requires: \usepackage{amsmath} % (and optionally \usepackage{enumitem} for spacing)
\begin{enumerate}[label=\textbf{\arabic*.}, leftmargin=*, itemsep=3pt, topsep=2pt]

  \item \textbf{Wheel travel.}  
  Converts encoder ticks into physical distance for each wheel.  
  Each tick represents a small rotation, and multiplying by $\frac{2\pi R}{T}$ gives the linear displacement:
  \[
    \Delta s_L=\frac{2\pi R}{T}\,\Delta N_L,\qquad
    \Delta s_R=\frac{2\pi R}{T}\,\Delta N_R
  \]

  \item \textbf{Differential motion.}  
  Computes how far the robot moved forward and how much it rotated.  
  The average wheel travel gives forward motion, while the difference gives rotation about the centerline:
  \[
    \Delta s=\frac{\Delta s_R+\Delta s_L}{2},\qquad
    \Delta\theta=\frac{\Delta s_R-\Delta s_L}{b}
  \]

  \item \textbf{Pose update (midpoint form).}  
  Updates global position $(x, y)$ and heading $\theta$ using the arc traced during motion.  
  The midpoint form accounts for the slight curvature of the path:
  \[
    x \leftarrow x + \Delta s \cos\!\left(\theta+\tfrac{\Delta\theta}{2}\right),\quad
    y \leftarrow y + \Delta s \sin\!\left(\theta+\tfrac{\Delta\theta}{2}\right),\quad
    \theta \leftarrow \mathrm{wrap}(\theta+\Delta\theta)
  \]
  This effectively “walks” the pose forward by one small motion step.

  \item \textbf{Exact arc formulation (optional).}  
  When the robot turns significantly, you can compute the exact circular arc rather than the midpoint approximation.  
  The instantaneous center of curvature (ICC) radius is:
  \[
    R_{\mathrm{icc}}=\frac{\Delta s}{\Delta\theta}
    =\frac{b}{2}\,\frac{\Delta s_R+\Delta s_L}{\Delta s_R-\Delta s_L},
  \]
  and the pose update follows the true circular trajectory:
  \[
    x \leftarrow x + R_{\mathrm{icc}}\,[\sin(\theta+\Delta\theta)-\sin\theta],\quad
    y \leftarrow y - R_{\mathrm{icc}}\,[\cos(\theta+\Delta\theta)-\cos\theta],\quad
    \theta \leftarrow \mathrm{wrap}(\theta+\Delta\theta)
  \]
  If $\Delta\theta\approx 0$ (straight motion), this simplifies to
  \[
    x \leftarrow x+\Delta s\cos\theta,\qquad
    y \leftarrow y+\Delta s\sin\theta.
  \]

  \item \textbf{Velocities (fixed timestep $\Delta t$).}  
  Converts incremental motion into linear and angular velocities.  
  These describe the robot’s instantaneous motion:
  \[
    v_L=\frac{2\pi R}{T}\frac{\Delta N_L}{\Delta t},\qquad
    v_R=\frac{2\pi R}{T}\frac{\Delta N_R}{\Delta t},\qquad
    v=\frac{v_R+v_L}{2},\quad
    \omega=\frac{v_R-v_L}{b}
  \]
  Integrating these gives continuous-time motion:
  \[
    \dot x=v\cos\theta,\qquad
    \dot y=v\sin\theta,\qquad
    \dot\theta=\omega.
  \]

\end{enumerate}

\textbf{Notes.}
\begin{enumerate}
  \item Ensure encoder signs are consistent so both wheels forward give $\Delta\theta\approx 0$.
  \item Include gearbox ratio and quadrature multiplication in $T$.
  \item Measure $b$ between effective wheel contact points; small errors bias $\theta$.
  \item Odometry drifts over time; fuse with IMU/vision for long runs.
\end{enumerate}

\end{document}

